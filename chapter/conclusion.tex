\chapter{结\texorpdfstring{\quad}{}论}

本文主要研究语言视觉激光多模态融合的机器人导航方法的研究及实现,研究方向是多模态融合的目标物体导航方法,并设计一个导航系统,在移动机器人上部署全局和局部路径规划算法,并在仿真环境和真实环境中进行目标物体导航实验。
	
本文的目标物体导航是在环境中根据多模态的数据到达预期的目标物体。现有的工作通常通过建图来标记环境中存在的目标位置,或是通过训练深度强化学习模型作为代理实时预测动作,以到达指定目标。但上述的方法无法通过视觉信息进行自主探索,忽略了激光雷达所获取的感知信息对于导航的约束和指导,从而导致系统导航成功率低、导航效率低和无法导航到目标半米内的问题。针对以上问题,提出一种多模态融合的目标物体导航方法,具体来说,该方法将导航任务拆分成全局路径规划和局部路径规划两个部分。在全局路径规划中,标记地图中的导航点,保留其位姿、图像、点云图和各点之间的拓扑信息,通过多模态融合网络得到各导航点与目标的匹配权值,结合dijkstra算法和方位优化算法规划出全局路径导航点序列。然后,在局部路径规划中,将多线激光与单目相机进行联合标定,结合目标检测、点云聚类和坐标变换方法得到目标具体位姿,发布导航任务以完成局部路径的规划,实现导航到目标半米内的闭环任务。Gazebo数据集上的实验表明,该方法在测试环境中优于最先进的方法,实验结果证明了该方法的有效性和效率。多模态融合的导航系统具有环境感知、目标认知、路径规划和自主导航四个方面的功能,可以完成真实环境下目标物体导航任务。

总的来说,本文的主要贡献如下:
\begin{enumerate}[topsep = 0 pt, itemsep= 0 pt, parsep=0pt, partopsep=0pt, leftmargin=44pt, itemindent=0pt, labelsep=6pt, label=(\arabic*)]
    \item 	本文提出了一种全局路径规划导航方法。与前人的工作相比,针对静态目标导航任务所提出的全局路径规划导航方法基于单目相机、激光雷达等多种传感器和基于多模态特征融合神经网络,增强系统对当前环境和导航过程中的认知和感知能力,再通过方位优化算法筛除噪声导航点,提高导航点选择的正确率的同时提高后续规划的计算响应速度,最后通过导航点规划算法加权融合多种策略进一步提高导航的准确率和导航效率。实验结果表明该方法具有一定的有效性和优越性。
    \item	本文提出了一种未知环境的目标物体探索方法。与前人的工作相比,针对动态目标导航任务所提出的局部目标物体探索方法基于多特征提取和融合的方法,在同一嵌入空间内利用注意力机制融合视觉特征和文本特征,有效的构建了视觉表示和目标物体所在导航方向的关联,使系统能够通过探索找到在变化的环境中的目标物体。
    \item	本文设计了一套单目相机和多线激光融合的图像点云融合方法,联合视觉观察的认知信息和多线点云的感知信息让移动机器人能够有效地在仿真环境和真实环境中依据自然语言指令完成目标导航任务,在移动机器人平台对所提出方法进行了测试加以验证。
\end{enumerate}



本文专注于研究一种适用于室内服务机器人的目标物体导航方法,并致力于构建一个可以部署在真实移动机器人上的语言视觉激光多模态融合的导航系统,使移动机器人既能根据自然语言指令进行全局路径规划,完成导航至目标半米内的任务。多模态融合导航系统具有环境认知与感知、路径规划和自主导航探索这三个主要方面的功能,可以完成室内真实环境下的多目标物体导航任务,该方法在多目标物体连续导航的仿真测试环境中优于最先进的方法,实验结果证明了该方法的有效性和效率。对于未来的展望,本文有以下三个方向可以进行考虑:
\begin{enumerate}[topsep = 0 pt, itemsep= 0 pt, parsep=0pt, partopsep=0pt, leftmargin=44pt, itemindent=0pt, labelsep=6pt, label=(\arabic*)]
    \item 	在真实环境的导航实验之中,我们采用激光雷达结合低功耗计算平台的导航架构在常规场景下表现稳定,但在复杂狭窄环境等受限空间内导航系统对误差源的容忍性下降,容易出现轨迹规划失效、运动控制失稳等问题。可以根据地图中障碍物与全局路径的几何关系标记狭窄环境并生成合适通行位姿对,在机器人出入被标记的狭窄环境时自动切换相应导航策略。通过这种全局成本地图膨胀化以规划更安全的全局路径的方法,机器人根据合适通行位姿分段规划全局路径以自适应环境
    \item	在真实环境的导航实验之中,由于真实环境和仿真环境的差异问题导致了局部路径规划中的特征提取、融合模型性能的下降,当移动机器人在存在行人、移动的椅子等动态的环境中进行导航时,局部探索方法会因遮挡物体的存在而发生碰撞从而导致导航的失败。可以通过在局部路径规划的特征提取模块加入视觉观察的深度图,帮助模型更精准地定位物体的位置信息、理解环境的拓扑结构和空间关系,进一步提高模型对于空间中的物体位置的感知,从而更高效、可靠地执行局部探索任务。
    \item	目标物体导航的环境中存在众多不同尺寸的物体,而多尺度特征可以增强模型的鲁棒性使其更具有抗干扰性,因此可以考虑添加多尺度特征可以更好地适应不同尺度下的物体。不同尺度的特征可以在不同的情况下提供更好的表征,让模型能够更好地应对光照变化、遮挡、旋转等因素的影响。另一方面,添加多尺度特征会增加计算复杂度,可以通过金字塔结构或多尺度卷积等方法共享计算从而提高效率。这种方法能够提供更丰富的语义信息、适应不同尺度的物体、增强模型的鲁棒性,并且能够更好地结合局部和全局信息,从而提高图像处理任务的性能和效率。 
    
\end{enumerate}


