\chapter{攻读博士/硕士学位期间取得的研究成果} %博士/硕士记得选其一
\pubfont % 论文撰写规范里,这章是5号宋体,\pubfont 设置字号为5号了。但其实很多论文用小四号也OK。
一、已发表(包括已接受待发表)的论文,以及已投稿、或已成文打算投稿、或拟成文投稿的论文情况\underline{\textbf{(只填写与学位论文内容相关的部分):}}
\begin{table}
	\centering{}%
	\pubfont 
	\begin{longtable}{|>{\centering}m{0.5cm}|m{1.8cm}|>{\centering}m{2.8cm}|>{\centering}m{2.5cm}|>{\centering}m{2.2cm}|>{\centering}m{2.cm}|>{\centering}m{1cm}|}
		\hline 
		\textbf{序号} & \textbf{作者(全体作者,按顺序排列)} & \textbf{题 目} 						   & \textbf{发表或投稿刊物名称、级别} & \textbf{发表的卷期、年月、页码} & \textbf{与学位论文哪一部分(章、节)相关} &\textbf{被索引收录情况}\tabularnewline
		\hline 
		1    & 毕盛,\textbf{杨礼铭},董敏,沈煜	 &  基于指令和多传感器融合的机器人室内导航方法 & 小型微型计算机系统,中文核心 & 已录用定稿 & 第三章部分内容 & CNKI \tabularnewline
		\hline 
		2	 & 董敏,谭皓禹,\textbf{杨礼铭},沈煜,陈章韶,毕盛	& 面向独居老人的智能居家监护系统	& 嵌入式技术与智能系统  & 2024年1(1) & 第四章部分内容 & Hans \tabularnewline
		\hline 
	\end{longtable}
\end{table}

二、与学位内容相关的其它成果(包括专利、著作、获奖项目等)
{\textbf{(只填写与学位论文内容相关的部分):}}
\begin{table}
	\centering{}%
	\pubfont 
	\begin{longtable}{|>{\centering}m{0.5cm}|m{1.8cm}|>{\centering}m{2.8cm}|>{\centering}m{3.0cm}|>{\centering}m{2.2cm}|>{\centering}m{2.5cm}|}
		\hline 
		\textbf{序号} & \textbf{作者(全体作者,按顺序排列)} & \textbf{题 目} 						   & \textbf{成果类型} & \textbf{状态} & \textbf{受理/登记时间}\tabularnewline
		\hline 
		1    & 谭皓禹;\textbf{杨礼铭};沈煜  & 面向独居老人的智能居家监护系统 & 中国研究生电子设计竞赛全国总决赛一等奖 & 已获取 & 2023-08-16   \tabularnewline
		\hline 
		2	 & 	毕盛;\textbf{杨礼铭};董敏 &  一种视觉与多线激光融合的移动机器人导航系统	 &  发明专利,专利号:CN202410023917.7 & 已公开 & 2024-04-12  \tabularnewline
		\hline 
	\end{longtable}
\end{table}


%注:这部分一言难尽,我努力了很久都没有把这个表做好。感觉学校给的这个表的模板非常反人类。看国外大学的博士论文,那种像参考文献著录信息那样一行一行的,比较美观。而这个框框很难放文字进去。

\normalsize % \normalsize可以将下文调回和正文一样的字号,这个随个人喜好。注释掉的话,致谢就就跟随《攻读博士/硕士学位期间取得的研究成果》的字号。